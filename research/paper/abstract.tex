one sentence introducing abusive content detection
\documentclass{12pt, a4paper}{article}

\title{Title Here}
\author{Andrew Galvin}

\begin{document}
\maketitle


The past decade has seen a rapid increase of user-generated data created from social media networks. Social media networks encourage users to communicate, create, and express their emotions to other's around the world. Inadvertently, it has created a new revolution of bullying, cyberbullying, as users can mask their identity and spread hate. Many state-of-the-art techniques utilize text classification to categorize content as abusive. However, with the increased usage of emojis to articulate a user's emotion a new avenue has been created for cyberbullying with them. In this paper, I explore different machine learning multi-class classification methods of detecting abusive content through data containing toxic Wikipedia's talk page from the Conversation AI team. In addition, I explore classifying toxic emojis based on the content of tweets gathered from Twitter.
